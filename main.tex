\documentclass{beamer}
\usepackage{hyperref}
\usepackage{ragged2e}

\begin{document}
%Information to be included in the title page:
\title{Relatório Parcial de Iniciação Científica (PUB)}
\subtitle{Regras fiscais: síntese e discussão sobre o arcabouço fiscal brasileiro}
\author{Cinthya Beneducci Afonso\\\small{Orientação: Fabiana Rocha}}
\institute{FEA-USP}
\date{2023}

\frame{\titlepage}

% Slide 2: Sumário
\begin{frame}
\frametitle{Sumário}

\begin{enumerate}
    \item Objetivos do projeto de pesquisa
    \item Problemática e motivações
    \begin{enumerate}
        \item O debate nacional e internacional sobre regras fiscais
        \item Base de dados do FMI
    \end{enumerate}
    \item Principais referências da literatura
    \begin{enumerate}
        \item Davoodi et al. (2022) e outras análises de estatística descritiva do FMI 
        \item O IRF da União Europeia e a metodologia de Budina, Schaechter e Kinda (2012) para o IRF et al. (2018) 
        \item HUANG e HO (2018) e Alberola et al. (2018)
    \end{enumerate}
    \item Progresso, próximos passos e conclusão
    \item Referências

\end{enumerate}

\end{frame}

%Slide 3: Objetivos do projeto de pesquisa
\begin{frame}
\frametitle{Objetivos do projeto de pesquisa}
    \begin{itemize}
        \item Fazer uso da série de dados temporal sobre regras fiscais do FMI para contribuir com análises que somem ao já realizado por Davoodi et al. (2022) e outros trabalhos da literatura
        \item Calcular e disponibilizar os valores ano-a-ano de um índice de força de regras fiscais para o Brasil 
        \item Estudar e analisar a literatura recente sobre as últimas propostas e alterações do arcabouço fiscal brasileiro
    
    \end{itemize}
\end{frame}

% Slide 4: O debate nacional e internacional sobre regras fiscais
\begin{frame}
\frametitle{Problemática e motivações}
\framesubtitle{O debate nacional e internacional sobre regras fiscais}
\begin{itemize}
    \item Discussão sobre arcabouços fiscais em pauta atualmente (ocorrência de diversos choques macroeconômicos recentes)
    \item Arcabouço fiscal brasileiro: propostas de alteração colocadas e novo arcabouço fiscal anunciado em mar/23
    \item Atualização em 2021 da série histórica de dados do FMI sobre as regras fiscais de 106 países
\end{itemize}

\end{frame}

% Slide 5: Base de dados do FMI
\begin{frame}
\frametitle{Problemática e motivações}
\framesubtitle{Base de dados do FMI}

\begin{itemize}
    \item Série histórica de dados do FMI sobre as regras fiscais de 106 países (última atualização: 2021)
    \item Variáveis categóricas (quase todas são dummies)
    \item Categoriza quatro tipos de regras fiscais: RR, ER, BBR e DR
    \item 5 informações quanto a cada tipo de regra:
    \begin{itemize}
        \item monitoramento por instituição extra-governamental 
        \item  procedimentos formais que incentivem o cumprimento da regra 
        \item nível de cobertura governamental 
        \item base legal
        \item existência de cláusula de escape 
    \end{itemize}
    \item Mais 6 informações sobre o arcabouço fiscal do país:
    \begin{itemize}
        \item existência de tetos plurianuais 
        \item instituição independente fazendo as presunções orçamentárias
        \item presença de LRFs
        \item presença de meta de balanço orçamentário estrutural/ciclicamente ajustado ou over-the-cycle 
        \item investimento público/outras prioridades excluídas das metas
    
    \end{itemize}
    \item Dados referentes ao Brasil desde 1998
\end{itemize}
    
\end{frame}


% Slide 6: Davoodi et al. e outras análises do FMI e a metodologia de Budina, Schaechter e Kinda (2012)
\begin{frame}
\frametitle{Principais referências da literatura}
\framesubtitle{Davoodi et al. (2022) e outras análises de estatística descritiva do FMI}
    
\end{frame}

% Slide 7: HUANG, Chiung-Ju; HO, Yuan-Hong (2018); Alberola (2018) e o IRF da União Europeia
\begin{frame}
\frametitle{Principais referências da literatura}
\framesubtitle{O IRF da União Europeia e a metodologia de Budina, Schaechter e Kinda (2012) para o IRF}

referencias
    
\end{frame}

\begin{frame}
\frametitle{Principais referências da literatura}
\framesubtitle{HUANG e HO (2018) e Alberola et al. (2018) }

referencias
    
\end{frame}
% Slide 8: Progresso, próximos passos e conclusão
\begin{frame}
\frametitle{Progresso, próximos passos e conclusão}
\begin{itemize}
    \item \justifying Progresso alcançado
    \begin{itemize}
        \item Revisão da literatura
        \item Estudo das principais propostas colocadas recentemente para um novo arcabouço brasileiro
        \item Identificação do que já foi desenvolvido na literatura
        \item Identificação de brechas para contribuição
        \item Preparação do arcabouço teórico necessário para efetuar as análises quantitativas na segunda parte da pesquisa
        \item Habilidades desenvolvidas: pesquisa acadêmica, programação em Python, oportunidade para usar \LaTeX
    \end{itemize}
    \item Próximos passos
    \begin{itemize}
        \item Produzir análises de estatística descritiva que somem ao já produzido em outros \emph{papers}
        \item Produzir o IRF para o Brasil a partir dos dados e metodologia disponíveis
        \item Explorar e compilar os debates recentes sobre o arcabouço fiscal brasileiro
        \item Experimentar mudanças no IRF brasileiro a partir de diferentes configurações fiscais
  
    \end{itemize}
\end{itemize}

    
\end{frame}

% Slide 9: Referências
\begin{frame}
\frametitle{Referências}

\begin{itemize}
    \item \small{ALBEROLA, Enrique et al. \textbf{Fiscal policy and the cycle in Latin America: the role of financing conditions and fiscal rules}. Ensayos sobre politica economica, v. 36, n. 85, p. 101-116, 2018.}
    \item BUDINA, Nina T.; SCHAECHTER, Andrea; KINDA, Tidiane. \textbf{Fiscal rules in response to the crisis: Toward the "next-generation" rules: A new dataset}. IMF working papers, v. 2012, n. 187, 2012.
    \item DAVOODI, Hamid R. et al. \textbf{Fiscal Rules and Fiscal Councils: Recent Trends and Performance during the COVID-19 Pandemic}. IMF Working Papers, v. 2022, n. 011, 2022.
    \item Directorate General for Economic and Financial Affairs. (2022). \textbf{Design of numerical fiscal rules 2021 [banco de dados]}. Disponível em: \url{https://economy-finance.ec.europa.eu/economic-research-and-databases/economic-databases/fiscal-governance-database_en}. Acesso em 5 de abril de 2023.
    \item HUANG, Chiung-Ju; HO, Yuan-Hong. \textbf{The Effectiveness of National Fiscal Rules in the Asia-Pacific Countries}. International Journal of Economics and Management Engineering, v. 12, n. 11, p. 1505-1509, 2018.
\end{itemize}


\end{frame}

\end{document}